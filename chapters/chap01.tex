\chapter{绪论}

1978年,英国牧师约翰·米歇尔在写给卡文迪什的信中提出了一种超大质量的天体,以至于光都无法逃离\citep{michell1784vii}。他还指出可以通过观察其对附近可见物体的引力效应来观测这种天体。但是在19世纪后,光的波动说成为主流以后,对于这种大质量天体的研究热情就减弱了。并且,由于光的速度恒定为光速,米歇尔将这一过程类比与向上抛的小球的想法是错误的。现在,我们已经揭开了一部分这个大质量天体——黑洞的神秘面纱,甚至获得了它的彩色照片\citep{akiyama2021first}。

\section{恒星的演化过程}
球对称不带电恒星,在质量满足一定条件的情况下(奥本海默-沃尔科夫极限),将会形成史瓦西黑洞。恒星在聚变过程中辐射能量,核心部分的收缩,挤压并加热了核心部分,将会点燃下一次聚变,氢$\rightarrow$氦$\rightarrow$碳$\rightarrow$氧等过程。而恒星的外层随之扩大,恒星变为了红巨星。而聚变的过程将会继续,一直到铁元素。由于铁元素的比结合能最高,聚变无法继续下去。恒星的内核开始冷却,并由于自身的引力而收缩。

质量约等于太阳质量且转动较慢的星体会变成一种高密度的物质状态白矮星。如果白矮星的质量大于钱德拉塞卡极限,那么引力将压倒电子简并压使白矮星进一步压缩,形成中子星。如果恒星的质量更大,以至于中子星也无法保持稳定,将会发生引力坍缩,如果坍缩是球对称的,最后将形成史瓦西黑洞。

\section{史瓦西黑洞}
1915年,爱因斯坦提出了他的广义相对论理论,显示了引力确实会对光的运动产生影响。几个月后史瓦西就提出了真空爱因斯坦方程的静态球对称解\citep{schwarzschild1916gravitationsfeld},在物理上描述一个球对称引力场,伯克霍夫证明了真空爱因斯坦方程的球对称解必为史瓦西度规\citep{birkhoff1923relativity}。史瓦西解的形式如下:
\begin{equation}
    ds^2=-\left(1-\frac{R_s}{r}\right)dt^2+\left(1-\frac{R_s}{r}\right)dr^2+r^2d\Omega^2
\end{equation}
其中$R_s=2GM/c^2$为史瓦西半径。时空在远离黑洞的情况下为渐近平坦时空。度规在史瓦西半径处的奇异性,可以通过坐标变换来去除。史瓦西坐标不适合描述粒子在$r<R_s$中的运动。由于时空流形独立存在,不依赖与坐标卡的选择,我们可以选择其他坐标系。

克鲁斯卡时空为史瓦西时空的最大延拓,选取克鲁斯卡坐标$\left(\upsilon, u, \theta, \phi\right)$,在这组坐标下度规没有奇点
\begin{equation}
    ds^2=\frac{32\mu^2}{r} e^{-r/2\mu}\left(-d\upsilon^2+du^2 \right)+r^2d\Omega^2
\end{equation}
其中$r$由式\eqref{eq: shawaxir}定义
\begin{equation}\label{eq: shawaxir}
    \left(u^2-\upsilon^2\right)=\left(\frac{r}{R_s}-1\right)e^{r/R_s}
\end{equation}

上文提到如果坍缩是球对称的,那么形成的黑洞是史瓦西黑洞,随之而来的问题是,是否可能在坍缩过程中存在小的非对称性扰动,从而事件视界不能形成。下一节我们会介绍奇点定理保证了黑洞会在自然界中形成。

\section{奇点和奇点定理}
时空奇点主要有两种定义。第一种是曲率无穷大的点或者区域。第二种是时空流形的测地线不完备性,严格的数学定义参考文献\citep{penrose1999question}。
奇点定理其中一个表述如下\citep{陈斌2018广义相对论}

\begin{description}
    \item[奇点定理]  令$M$是一个具有一般性度规$g_{\mu \nu}$的时空流形,度规满足爱因斯坦方程且相关物质的能动张量满足强能量条件。如果在$M$中有一个陷俘面,则流形上必然有一个闭合的类时曲线或者一个奇点(体现为非完备的类时或类光测地线)。
\end{description}

奇点附近的量子效应应当非常剧烈,在描述量子引力的弦理论中,奇点应当是不存在的。但是,由于我们还没有一个完整的量子引力工作理论,所以不能保证量子效应将解决时空奇点。事实上,量子效应可能在时空奇点附近被抑制,伸缩子带电G-H-S黑洞的弦耦合在奇点处变得很弱,这表明它的行为是“经典的”\citep{ong2020space}。

所以,我们不能简单的希望量子引力会解决奇点,尤其是会给广义相对论带来不确定性的裸奇点。为了避免裸奇点的产生,彭索斯提出宇宙监督假设。宇宙监督假设有两个版本,我们将在下一节详细介绍。

\section{宇宙监督假设}
史瓦西黑洞这样的类空奇点并不是问题,因为任何内部的观察者都看不到奇点,在视界内部,径向方向变味了时间方向,所以奇点存在于观察者的未来,奇点无法传递任何信息,正如我们不能发短信跟过去的自己。
\section{弱宇宙监督假设}
弱宇宙监督假设要求,奇点应在黑洞视界之后。根据黑洞的定义,关于奇点的信息不能影响外部世界。

Choptuik证明了在非常特殊的初始条件下\citep{choptuik1993universality},可以形成一个裸奇点。但这样的粒子已经被弱宇宙监督假设排除,因为弱宇宙监督假设只要求在一般的初始条件下不形成裸奇点。

另一个违反弱宇宙监督假设的尝试为摧毁一个黑洞的视界,这也是本文采用的方法。Wald的论文\citep{wald1974gedanken}和之后的许多工作都证明了当黑洞越来越接近极值时,也就越来越难扔进可能破坏视界的粒子。

但是弱宇宙监督假设并不是那么牢不可破了。Santos 和他的合作者对于边界有特殊化学势的黑洞—Hovering 黑洞进行了数值分析\citep{horowitz2015hovering,crisford2017violating},似乎得出了和 Wald 相反的结论。
\subsection{思想实验}
向极端黑洞中丢入一个测试粒子,使黑洞捕获其后最终的物体不再满足黑洞存在的条件,从而检验是否导致视界被破坏。由于只需要计算出黑洞的存在条件、粒子能够进入黑洞的条件和破坏视界时粒子需要满足的条件,这种方法在计算上非常简单易行。

后来有研究称,如果 Wald 使用的测试粒子电荷满足一定的大小,那么在只考虑一阶修正项的情况下,是可以破坏黑洞视界的\citep{hubeny1999overcharging}。但是很快就被发现,在此情况下测试粒子对背景时空的影响不能忽略\citep{hod2002cosmic,barausse2010test,colleoni2015overspinning,wald2018kerr,sorce2017gedanken},弱宇宙监督假设依然得到了保证。

\section{强宇宙监督假设}
强宇宙监督假设本质上要求未来应该是完全确定的,即使是在黑洞里。毕竟,一个观测者可以被送入一个任意大小的黑洞,如果他或她在旅行中幸存下来,就可以在黑洞内进行物理观测和实验。至少在原则上,人们可以在黑洞内部进行物理研究,而不必向外部时空的同事报告研究结果\citep{ong2020space}。其中一种简单的表述为真实时空是整体双曲时空,即时空中不应当存在柯西视界。

\section{本文的主要工作}

在第二章,我们通过思想实验的方法,尝试让黑洞捕获一个可以摧毁黑洞视界的测试粒子,验证了在渐近反德西特时空中黑洞的例子,分别为莱斯特-努德斯特伦黑洞、双曲带电黑洞和爱因斯坦-高斯-博尼特引力模型中的黑洞,其中对于双曲带电黑洞分为正质量和负质量情形。最终发现,由于静电斥力的影响,可以破坏黑洞视界的测试粒子总是不能被捕获。