\chapter{绪论}

1978年,英国牧师约翰·米歇尔在写给卡文迪什的信中提出了一种超大质量的天体,以至于光都无法逃离\citep{michell1784vii}。他还指出可以通过观察其对附近可见物体的引力效应来观测这种天体。但是在19世纪后,光的波动说成为主流以后,对于这种大质量天体的研究热情就减弱了。并且,由于光的速度恒定为光速,米歇尔将这一过程类比与向上抛的小球的想法是错误的。现在,我们已经揭开了一部分这个大质量天体的神秘面纱,甚至获得了它的彩色照片\citep{akiyama2021first}。

\section{恒星的演化过程}
球对称不带电恒星,在质量满足一定条件的情况下(奥本海默-沃尔科夫极限),将会形成史瓦西黑洞。恒星在聚变过程中辐射能量,核心部分的收缩,挤压并加热了核心部分,将会点燃下一次聚变,氢$\rightarrow$氦$\rightarrow$碳$\rightarrow$氧等过程。而恒星的外层随之扩大,恒星变为了红巨星。而聚变的过程将会继续,一直到铁元素。由于铁元素的比结合能最高,聚变无法继续下去。恒星的内核开始冷却,并由于自身的引力而收缩。

质量约等于太阳质量且转动较慢的星体会变成一种高密度的物质状态白矮星。如果白矮星的质量大于钱德拉塞卡极限,那么引力将压倒电子简并压使白矮星进一步压缩,形成中子星。如果很行的质量更大中子星也无法保持稳定,将会发生引力坍缩,如果坍缩是球对称的,最后将形成史瓦西黑洞。

\section{史瓦西黑洞}
1915年,爱因斯坦提出了他的广义相对论理论,显示了引力确实会对光的运动产生影响。几个月后史瓦西就提出了真空爱因斯坦方程的静态球对称解\citep{schwarzschild1916gravitationsfeld},在物理上描述一个球对称引力场,伯克霍夫证明了真空爱因斯坦方程的球对称解必为史瓦西度规\citep{birkhoff1923relativity}。史瓦西解的形式如下:
\begin{equation}
    ds^2=-\left(1-\frac{R_s}{r}\right)dt^2+\left(1-\frac{R_s}{r}\right)dr^2+r^2d\Omega^2
\end{equation}
其中$R_s=2GM/c^2$为史瓦西半径。时空在远离黑洞的情况下为渐进平坦时空。度规在史瓦西半径处的奇异性,可以通过坐标变换来去除。史瓦西坐标不适合描述粒子在$r<R_s$中的运动。由于时空流形独立存在,不依赖与坐标卡的选择,我们可以选择其他坐标系。

\subsection{克鲁斯卡时空}
克鲁斯卡时空为史瓦西时空的最大延拓,选取克鲁斯卡坐标$\left(\upsilon, u, \theta, \phi\right)$,在这组坐标下度规没有奇点
\begin{equation}
    ds^2=\frac{32\mu^2}{r} e^{-r/2\mu}\left(-d\upsilon^2+du^2 \right)+r^2d\Omega^2
\end{equation}
其中$r$由式\eqref{eq: shawaxir}定义
\begin{equation}\label{eq: shawaxir}
    \left(u^2-\upsilon^2\right)=\left(\frac{r}{R_s}-1\right)e^{r/R_s}
\end{equation}

上文提到如果坍缩是球对称的,那么形成的黑洞是史瓦西黑洞,随之而来的问题是,是否可能在坍缩过程中存在小的非对称性扰动,从而事件视界不能形成。下一节我们会介绍奇点定理保证了黑洞会在自然界中形成。

\section{奇点和奇点定理}
时空奇点主要有两种类型。第一种是曲率无穷大的点或者区域。第二类是时空流形的测地线不完备性,严格的数学定义参考文献\citep{penrose1999question}。
\subsection{奇点定理}
奇点定理其中一个表述如下\citep{陈斌2018广义相对论}

\begin{description}
    \item[奇点定理]  令$M$是一个具有一般性度规$g_{\mu \nu}$的时空流形,度规满足爱因斯坦方程且相关物质的能动张量满足强能量条件。如果在$M$中有一个捕获面,则流形上必然有一个闭的类时曲线或者一个奇点(体现为非完备的类时或类光测地线)
\end{description}

宇宙学奇点附近的量子效应应当非常剧烈,在描述量子引力的弦理论中,宇宙学奇点是不存在的。但是,由于我们还没有一个完整的量子引力工作理论,所以根本不能保证量子效应将解决时空奇点。事实上,量子效应可能在时空奇点附近被抑制膨胀电荷G-H-S黑洞的弦耦合在奇点处变得很弱,这表明它的行为是“经典的”\citep{ong2020space}。

所以,我们不能简单的希望量子引力会解决奇点,尤其是会给广义相对论带来不确定性的裸奇点。为了避免裸奇点的产生宇宙监督假设被彭索斯提出。宇宙监督假设有两个版本,我们将在下一章详细介绍。