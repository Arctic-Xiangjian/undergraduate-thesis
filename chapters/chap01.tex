\chapter{引言}
1978年,英国牧师约翰·米歇尔在写给卡文迪什的信中提出了一种超大质量的天体,以至于光都无法逃离\citep{michell1784vii}。他还指出可以通过观察其对附近可见物体的引力效应来观测这种天体。但是在19世纪后,光的波动说成为主流以后,对于这种大质量天体的研究热情就减弱了。并且,由于光的速度恒定为光速,米歇尔将这一过程类比与向上抛的小球的想法是错误的。

1915年,爱因斯坦提出了他的广义相对论理论,显示了引力确实会对光的运动产生影响。几个月后史瓦西就提出了真空爱因斯坦方程的静态球对称解\citep{schwarzschild1916gravitationsfeld},在物理上描述一个球对称引力场,伯克霍夫证明了真空爱因斯坦方程的球对称解必为史瓦西度规\citep{birkhoff1923relativity}。史瓦西解的形式如下:
\begin{equation}
    ds^2=-\left(1-\frac{R_s}{r}\right)dt^2+\left(1-\frac{R_s}{r}\right)dr^2+r^2d\Omega^2
\end{equation}
其中$R_s=2GM/c^2$为史瓦西半径。时空在远离黑洞的情况下为渐进平坦时空。度规在史瓦西半径处的奇异性,可以通过坐标变换来去除。
\section{奇点和奇点定理}
\subsection{奇点导致的问题}
\section{黑洞}
\subsection{黑洞的形成}
\subsection{史瓦西黑洞}

\subsection{事件视界}

