\chapter{反德西特时空下的黑洞}
\section{德西特时空和反德西特时空}
\section{莱斯纳-努德斯特伦解}
史瓦西时空描述的为真空静态球对称星体外部的时空,其带电推广为莱斯纳-努德斯特伦时空。相较于史瓦西时空,它存在多个黑洞视界,存在着类时的奇点。

带电球对称星体外的作用量可以写为:
\begin{equation}
    S=\frac{1}{16 \pi G_D}\int d^D x \sqrt{-g}\left(R-\frac{g_{0}^2}{4 \pi}F^{\mu \nu}F_{\mu \nu}\right)
\end{equation}
可以得到爱因斯坦方程为
\begin{equation}
    R_{\mu \nu}-\frac{1}{2}g_{\mu \nu}R=8\pi G_D g_0^2 T^{\left(F\right)}_{\mu \nu}
\end{equation}
其中
\begin{equation}
    T^{\left(F\right)}_{\mu \nu}=\frac{1}{4\pi}\left(F_{\mu \rho }F\indices{_{\nu}^{\rho}}-\frac{1}{4}g_{\mu \nu}F_{\rho \sigma}F^{\rho \sigma} \right)
\end{equation}
由此,我们可以得到莱斯纳-努德斯特伦度规\citep{陈斌2018广义相对论}
\begin{equation}\label{eq: RNmetric}
     ds^2=-\left(1-\frac{2M}{r}+\frac{Q^2}{r^2}\right)dt^2+\left(1-\frac{2M}{r}+\frac{Q^2}{r^2}\right)^{-1}dr^2+r^2\left(d\theta^2+\sin ^2 \theta d\phi^2\right)
\end{equation}
它描述了静态球对称带电荷星体的外部时空,同史瓦西解类似,我们希望能够描述整个时空的情况。除了$r=0$这一个时空奇性以外,若$1-2M / r+Q^2 / r^2=0$有解,那么还有其他奇性存在。简单求解发现$r$值满足
\begin{equation}\label{eq: RNhorizon}
    r_{\pm}=M \pm \sqrt{M^2-Q^2}
\end{equation}
