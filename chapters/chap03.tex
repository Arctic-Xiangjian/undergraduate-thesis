\chapter{反德西特时空下的黑洞}
\section{德西特时空和反德西特时空}
\subsection{常曲率空间}
\subsection{德西特时空}
\subsection{反德西特时空}
\section{莱斯纳-努德斯特伦解}
史瓦西时空描述的为真空静态球对称星体外部的时空,其带电推广为莱斯纳-努德斯特伦时空。相较于史瓦西时空,它存在多个黑洞视界,存在着类时的奇点。

带电球对称星体外的作用量可以写为:
\begin{equation}
    S=\frac{1}{16 \pi G_D}\int d^D x \sqrt{-g}\left(R-\frac{g_{0}^2}{4 \pi}F^{\mu \nu}F_{\mu \nu}\right)
\end{equation}
可以得到爱因斯坦方程为
\begin{equation}
    R_{\mu \nu}-\frac{1}{2}g_{\mu \nu}R=8\pi G_D g_0^2 T^{\left(F\right)}_{\mu \nu}
\end{equation}
其中
\begin{equation}
    T^{\left(F\right)}_{\mu \nu}=\frac{1}{4\pi}\left(F_{\mu \rho }F\indices{_{\nu}^{\rho}}-\frac{1}{4}g_{\mu \nu}F_{\rho \sigma}F^{\rho \sigma} \right)
\end{equation}
由此,我们可以得到莱斯纳-努德斯特伦度规\citep{陈斌2018广义相对论}
\begin{equation}\label{eq: RNmetric}
     ds^2=-\left(1-\frac{2M}{r}+\frac{Q^2}{r^2}\right)dt^2+\left(1-\frac{2M}{r}+\frac{Q^2}{r^2}\right)^{-1}dr^2+r^2\left(d\theta^2+\sin ^2 \theta d\phi^2\right)
\end{equation}
它描述了静态球对称带电荷星体的外部时空,同史瓦西解类似,莱斯纳-努德斯特伦解是爱因斯坦-麦克斯韦理论的唯一球对称解。当$Q=P=0$,既电荷和磁荷都为零时,莱斯纳-努德斯特伦解退化为史瓦西解。

如果要描述整个时空的情况,除了$r=0$这一个时空奇性以外,若$1-2M / r+Q^2 / r^2=0$有解,那么还有其他奇性存在。简单求解发现$r$值满足
\begin{equation}\label{eq: RNhorizon}
    r_{\pm}=M \pm \sqrt{M^2-Q^2}
\end{equation}
根据解的形式,分三种情况讨论
\begin{enumerate}
    \item 若$M^2<Q^2$,此时$r$无实数值,这种情况下莱斯纳-努德斯特伦时空只有一个奇性$r=0$,在此情况下莱斯纳-努德斯特伦黑洞也不能存在;
    \item 若$M^2>Q^2$,此时除$r=0$以外,$r_{\pm}$也为奇性,它们为坐标奇性\citep{梁灿彬2006微分几何入门与广义相对论},$r_+$被称为外视界,$r_-$被称为内视界。在外视界$r_+$以外,均有球对称性,存在类时的基灵矢量;
    \item 若$M^2=Q^2$,此时上述两个奇性合二为一,这种情况被称为极端莱斯纳-努德斯特伦黑洞。第二章所提到的Wald的测试粒子方法\citep{wald1974gedanken},所采用的就是此种情况下的莱斯纳-努德斯特伦黑洞来验证弱宇宙监督假设。
\end{enumerate}
\subsection{反德西特时空下的莱斯纳-努德斯特伦黑洞}
式\eqref{eq: RNmetric}给出了零曲率空间下的莱斯纳-努德斯特伦黑洞的度规。在反德西特时空中,莱斯纳-努德斯特伦黑洞的度规为
\begin{equation}\label{eq: RN-AdSmetric}
    ds^2=-f\left(r\right)dt^2+\frac{1}{f\left(r\right)}dr^2+r^2\left(d\theta^2+\sin^2\phi d\phi ^2\right)
\end{equation}
度规的逆
\begin{equation}\label{eq: RN-AdSinversmetric}
    \left(\frac{\partial }{\partial s}\right)^2=-\frac{1}{f\left(r\right)}\left(\frac{\partial }{\partial t}\right)^2+f\left(r\right)\left(\frac{\partial }{\partial r}\right)^2+\frac{1}{r^2}\left(\left(\frac{\partial }{\partial \theta}\right)^2+\frac{1}{\sin ^2 \phi}\left(\frac{\partial }{\partial \phi}\right)^2\right)
\end{equation}
其中$f\left(r\right)$
\begin{equation}\label{eq: f(r)}
    f\left(r\right)=1-\frac{2M}{r}+\frac{Q^2}{r^2}+\frac{r^2}{l^2}
\end{equation}
其中$l$反德西特半径,它和宇宙学常数的关系为$\Lambda=-3/l^2$,参数$M$和$Q$,可以被理解为黑洞的质量和黑洞携带和电荷。
规范势可以写为
\begin{equation}\label{eq: vectorpotion}
    A=A_t\left(r\right)dt=-\frac{Q}{r}dt
\end{equation}

采用测试粒子的方法来验证对于反德西特时空下的莱斯纳-努德斯特伦黑洞弱宇宙监督假设是否成立,首先需要根据反德西特时空下的莱斯纳-努德斯特伦黑洞的度规式\eqref{eq: RN-AdSmetric},来讨论出极端反德西特时空下的莱斯纳-努德斯特伦黑洞的视界半径,此时令$ f(r)=0 $,会发现和常曲率空间下的莱斯纳-努德斯特伦黑洞不同的是,由于$r^2/l^2$这一项的存在,$ f(r)=0 $事实上为四次方程,这将带来四个根的存在。这并不说明反德西特时空下的莱斯纳-努德斯特伦黑洞存在四个视界半径,通过讨论发现,其中两根的值为虚数没有物理意义。反德西特时空下的莱斯纳-努德斯特伦黑洞的视界半径为
\begin{align}
    r_+=r + r_* \label{eq: rnr+} \\
    r_-=r-r_* \label{eq: rnr-} 
\end{align}
其中
\begin{align*}
    &r = \frac{1}{2} \gamma \\
    &r_* = \frac{1}{2} \sqrt{(\frac{4 l^2
    M}{\gamma}-\gamma^2-\frac{2 l^2}{3})}  \\
    &\alpha,\ \beta,\ \gamma \text{的值分别为} \\
    &\alpha =12 l^2 Q^2+l^4 \\ 
    &\beta =2 l^6 + 108 l^4 M^2 - 72 l^4 Q^2 \\
    &\gamma = \sqrt{\frac{\sqrt[3]{\sqrt{\beta^2-4 \alpha^3}+\beta}}{3\sqrt[3]{2}}+\frac{\sqrt[3]{2} \alpha}{3 \sqrt[3]{\sqrt{\beta^2-4\alpha^3}+\beta}}-\frac{2 l^2}{3}} 
\end{align*}
\subsubsection{反德西特时空下的莱斯纳-努德斯特伦黑洞存在条件}
观察解的形式发现,对比零曲率空间下的莱斯纳-努德斯特伦黑洞,得到反德西特时空下的极端莱斯纳-努德斯特伦黑洞为$r_+=r_-$,反德西特时空下的莱斯纳-努德斯特伦黑洞存在条件应为
\begin{equation}\label{eq: rnadsconuneq}
    \beta^2-4\alpha^3 \geq 0
\end{equation}
求解式\eqref{eq: rnadsconuneq},得到关于质量参数$M$、电荷参数$Q$和反德西特半径$l$的关系式\eqref{eq: rnadscondition}
\begin{equation}\label{eq: rnadscondition}
    M \geq \frac{\sqrt{\delta } (\delta +3) l}{3 \sqrt{6}} \qquad\qquad  \delta= \sqrt{\frac{12 Q^2}{l^2}+1}-1 
\end{equation}
还需要式\eqref{eq: rnadscondition}使得式\eqref{eq: suibian}得到满足
\begin{equation}\label{eq: suibian}
    \frac{\sqrt[3]{\sqrt{\beta^2-4 \alpha^3}+\beta}}{3\sqrt[3]{2}}+\frac{\sqrt[3]{2} \alpha}{3 \sqrt[3]{\sqrt{\beta^2-4\alpha^3}+\beta}}-\frac{2 l^2}{3} \geq 0
\end{equation}
简单计算验证,发现式\eqref{eq: suibian}得到满足。

以上的讨论是为了获得在反德西特时空下的极端莱斯纳-努德斯特伦黑洞的信息,通过比较前文中零曲率空间下的莱斯纳-努德斯特伦黑洞,我们可以知道反德西特时空下的极端莱斯纳-努德斯特伦黑洞的视界半径,将式\eqref{eq: rnadsconofmass}代入式\eqref{eq: rnr+}和式\eqref{eq: rnr-}
\begin{equation}\label{eq: rnadsconofmass}
    M=\frac{\sqrt{\delta } (\delta +3) l}{3 \sqrt{6}}
\end{equation}
反德西特时空下的极端莱斯纳-努德斯特伦黑洞的视界半径为
\begin{equation}\label{eq: rnadshorizonex}
    r=r_+=r_-=l\sqrt{\frac{\delta}{6}} 
\end{equation}

尝试创造一个违反式\eqref{eq: rnadscondition}的时空,通过想黑洞中丢入一个满足特殊条件的粒子,可以让式\eqref{eq: rnadscondition}取等号。假设此黑洞捕获了一个有着能量$E$,电荷为$e$,无自旋的粒子。因此,最后的物体具有电荷$\left(e+Q\right)$,质量不大于$\left(m+E\right)$。测试粒子需要满足$e\ll Q$,$E\ll M$的条件。简单计算得出,若末态时空不能保持为反德西特时空下的极端莱斯纳-努德斯特伦时空,那么式\eqref{eq: rnadscondition}成为
\begin{equation}\label{eq: rnadstestcon}
    M+E < \frac{\sqrt{\delta } (\delta +3) l}{3 \sqrt{6}} \quad\quad\quad\quad  \delta= \sqrt{\frac{12 \left(Q+e\right)^2}{l^2}+1}-1
\end{equation}
由于$e\ll Q$,$E\ll M$,式\eqref{eq: rnadstestcon}平方,泰勒展开,得到如果测试粒子破坏反德西特时空下的极端莱斯纳-努德斯特伦黑洞$E$需要满足的条件
\begin{equation}\label{eq: rnadsdiscon}
    E < \frac{Q e \left(3+\delta\right)}{3M} \qquad\qquad  \delta= \sqrt{\frac{12 Q^2}{l^2}+1}-1
\end{equation}

本次思想实验的目标就是,尝试让黑洞捕获一个满足式\eqref{eq: rnadsdiscon}条件的小物体。如此,最终的时空必然会出现违反弱宇宙监督假设的情况\citep{wald1974gedanken}。
\subsubsection{测试粒子的轨道}
Carter非常详尽的研究了在克尔-纽曼几何下的粒子运动\cite{carter1968global}。有着静止质量$M$和电荷$Q$的带电粒子的运动方程为
\begin{equation}
    \frac{D^2x^\mu}{D s^2}=\frac{d^2x^\mu}{ds^2}+\Gamma^{\mu}_{\rho\sigma }\frac{dx^\rho}{ds}\frac{dx^\sigma}{ds}=\frac{e}{m}F^{\mu\nu }\frac{dx_\nu}{ds}
\end{equation}
粒子的能量$E$和沿对称轴的角动量$L$分量由以下给出,由Wald的方法\cite{wald1974gedanken},根据式()
\begin{align}
    -E&=p_t=\frac{\partial \mathcal{L}}{\partial \dot{t}}=Mg_{tt}\frac{dx^t}{ds}+QA_t \\
    L&=p_\phi=\frac{\partial \mathcal{L}}{\partial \dot{\phi}}=0
\end{align}

在之前的讨论中,我们现在试图找到一个可以进入反德西特时空下的极端莱斯纳-努德斯特伦黑洞的粒子轨道,并且满足式\eqref{eq: rnadsdiscon}的条件。
