\chapter{反德西特时空下的黑洞}
\section{德西特时空和反德西特时空}
\subsection{常曲率空间}
\subsection{德西特时空}
\subsection{反德西特时空}
\section{莱斯纳-努德斯特伦解}
史瓦西时空描述的为真空静态球对称星体外部的时空,其带电推广为莱斯纳-努德斯特伦时空。相较于史瓦西时空,它存在多个黑洞视界,存在着类时的奇点。

带电球对称星体外的作用量可以写为:
\begin{equation}
    S=\frac{1}{16 \pi G_D}\int d^D x \sqrt{-g}\left(R-\frac{g_{0}^2}{4 \pi}F^{\mu \nu}F_{\mu \nu}\right)
\end{equation}
可以得到爱因斯坦方程为
\begin{equation}
    R_{\mu \nu}-\frac{1}{2}g_{\mu \nu}R=8\pi G_D g_0^2 T^{\left(F\right)}_{\mu \nu}
\end{equation}
其中
\begin{equation}
    T^{\left(F\right)}_{\mu \nu}=\frac{1}{4\pi}\left(F_{\mu \rho }F\indices{_{\nu}^{\rho}}-\frac{1}{4}g_{\mu \nu}F_{\rho \sigma}F^{\rho \sigma} \right)
\end{equation}
由此,我们可以得到莱斯纳-努德斯特伦度规\citep{陈斌2018广义相对论}
\begin{equation}\label{eq: RNmetric}
     ds^2=-\left(1-\frac{2M}{r}+\frac{Q^2}{r^2}\right)dt^2+\left(1-\frac{2M}{r}+\frac{Q^2}{r^2}\right)^{-1}dr^2+r^2\left(d\theta^2+\sin ^2 \theta d\phi^2\right)
\end{equation}
它描述了静态球对称带电荷星体的外部时空,同史瓦西解类似,莱斯纳-努德斯特伦解是爱因斯坦-麦克斯韦理论的唯一球对称解。当$Q=P=0$,既电荷和磁荷都为零时,莱斯纳-努德斯特伦解退化为史瓦西解。

如果要描述整个时空的情况,除了$r=0$这一个时空奇性以外,若$1-2M / r+Q^2 / r^2=0$有解,那么还有其他奇性存在。简单求解发现$r$值满足
\begin{equation}\label{eq: RNhorizon}
    r_{\pm}=M \pm \sqrt{M^2-Q^2}
\end{equation}
根据解的形式,分三种情况讨论
\begin{enumerate}
    \item 若$M^2<Q^2$,此时$r$无实数值,这种情况下莱斯纳-努德斯特伦时空只有一个奇性$r=0$,在此情况下莱斯纳-努德斯特伦黑洞也不能存在;
    \item 若$M^2>Q^2$,此时除$r=0$以外,$r_{\pm}$也为奇性,它们为坐标奇性\citep{梁灿彬2006微分几何入门与广义相对论},$r_+$被称为外视界,$r_-$被称为内视界。在外视界$r_+$以外,均有球对称性,存在类时的基灵矢量;
    \item 若$M^2=Q^2$,此时上述两个奇性合二为一,这种情况被称为极端莱斯纳-努德斯特伦黑洞。第二章所提到的Wald的测试粒子方法\citep{wald1974gedanken},所采用的就是此种情况下的莱斯纳-努德斯特伦黑洞来验证弱宇宙监督假设。
\end{enumerate}
\subsection{反德西特时空下的莱斯纳-努德斯特伦黑洞}
式\eqref{eq: RNmetric}给出了常曲率空间下的莱斯纳-努德斯特伦黑洞的度规。在反德西特时空中,莱斯纳-努德斯特伦黑洞的度规为
\begin{equation}\label{eq: RN-AdSmetric}
    ds^2=-f\left(r\right)dt^2+\frac{1}{f\left(r\right)}dr^2+r^2\left(d\theta^2+\sin^2\phi\right)
\end{equation}
其中$f\left(r\right)$
\begin{equation}\label{eq: f(r)}
    f\left(r\right)=1-\frac{2M}{r}+\frac{Q^2}{r^2}+\frac{r^2}{l^2}
\end{equation}
其中$l$反德西特半径,它和宇宙学常数的关系为$\Lambda=-3/l^2$,参数$M$和$Q$,可以被理解为黑洞的质量和黑洞携带和电荷。
规范势可以写为
\begin{equation}\label{eq: vectorpotion}
    A=A_t\left(r\right)dt=-\frac{Q}{r}dt
\end{equation}

采用测试粒子的方法来验证对于反德西特时空下的莱斯纳-努德斯特伦黑洞弱宇宙监督假设是否成立,首先需要根据反德西特时空下的莱斯纳-努德斯特伦黑洞的度规式\eqref{eq: RN-AdSmetric},来讨论出极端反德西特时空下的莱斯纳-努德斯特伦黑洞的视界半径,此时令$ f(r)=0 $,会发现和常曲率空间下的莱斯纳-努德斯特伦黑洞不同的是,由于$r^2/l^2$这一项的存在,$ f(r)=0 $事实上为四次方程,这将带来四个根的存在。这并不说明反德西特时空下的莱斯纳-努德斯特伦黑洞存在四个视界半径,通过讨论发现,其中两根的值为虚数没有物理意义。德西特时空下的莱斯纳-努德斯特伦黑洞的视界半径为
\begin{align}
    r_+=r + r_* \label{eq: rnr+} \\
    r_-=r-r_* \label{eq: rnr-} 
\end{align}
其中
\begin{align*}
    &r = \frac{1}{2} \gamma \\
    &r_* = \frac{1}{2} \sqrt{(\frac{4 l^2
    M}{\gamma}-\gamma^2-\frac{2 l^2}{3})}  \\
    &\alpha,\ \beta,\ \gamma \text{的值分别为} \\
    &\alpha =12 l^2 Q^2+l^4 \\ 
    &\beta =2 l^6 + 108 l^4 M^2 - 72 l^4 Q^2 \\
    &\gamma = \sqrt{\frac{\sqrt[3]{\sqrt{\beta^2-4 \alpha^3}+\beta}}{3\sqrt[3]{2}}+\frac{\sqrt[3]{2} \alpha}{3 \sqrt[3]{\sqrt{\beta^2-4\alpha^3}+\beta}}-\frac{2 l^2}{3}} 
\end{align*}
\subsubsection{极端黑洞情况}
观察解的形式发现,对比常曲率空间下的莱斯纳-努德斯特伦黑洞