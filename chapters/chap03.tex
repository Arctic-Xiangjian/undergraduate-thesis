\chapter{反德西特时空下的黑洞}
\section{德西特时空和反德西特时空}
\subsection{常曲率空间}
\subsection{德西特时空}
\subsection{反德西特时空}
\section{莱斯纳-努德斯特伦解}
史瓦西时空描述的为真空静态球对称星体外部的时空,其带电推广为莱斯纳-努德斯特伦时空。相较于史瓦西时空,它存在多个黑洞视界,存在着类时的奇点。

带电球对称星体外的作用量可以写为:
\begin{equation}
    S=\frac{1}{16 \pi G_D}\int d^D x \sqrt{-g}\left(R-\frac{g_{0}^2}{4 \pi}F^{\mu \nu}F_{\mu \nu}\right)
\end{equation}
可以得到爱因斯坦方程为
\begin{equation}
    R_{\mu \nu}-\frac{1}{2}g_{\mu \nu}R=8\pi G_D g_0^2 T^{\left(F\right)}_{\mu \nu}
\end{equation}
其中
\begin{equation}
    T^{\left(F\right)}_{\mu \nu}=\frac{1}{4\pi}\left(F_{\mu \rho }F\indices{_{\nu}^{\rho}}-\frac{1}{4}g_{\mu \nu}F_{\rho \sigma}F^{\rho \sigma} \right)
\end{equation}
由此,我们可以得到莱斯纳-努德斯特伦度规\citep{陈斌2018广义相对论}
\begin{equation}\label{eq: RNmetric}
     ds^2=-\left(1-\frac{2M}{r}+\frac{Q^2}{r^2}\right)dt^2+\left(1-\frac{2M}{r}+\frac{Q^2}{r^2}\right)^{-1}dr^2+r^2\left(d\theta^2+\sin ^2 \theta d\phi^2\right)
\end{equation}
它描述了静态球对称带电荷星体的外部时空,同史瓦西解类似,莱斯纳-努德斯特伦解是爱因斯坦-麦克斯韦理论的唯一球对称解。当$Q=P=0$,既电荷和磁荷都为零时,莱斯纳-努德斯特伦解退化为史瓦西解。

如果要描述整个时空的情况,除了$r=0$这一个时空奇性以外,若$1-2M / r+Q^2 / r^2=0$有解,那么还有其他奇性存在。简单求解发现$r$值满足
\begin{equation}\label{eq: RNhorizon}
    r_{\pm}=M \pm \sqrt{M^2-Q^2}
\end{equation}
根据解的形式,分三种情况讨论
\begin{enumerate}
    \item 若$M^2<Q^2$,此时$r$无实数值,这种情况下莱斯纳-努德斯特伦时空只有一个奇性$r=0$,在此情况下莱斯纳-努德斯特伦黑洞也不能存在;
    \item 若$M^2>Q^2$,此时除$r=0$以外,$r_{\pm}$也为奇性,它们为坐标奇性\citep{梁灿彬2006微分几何入门与广义相对论},$r_+$被称为外视界,$r_-$被称为内视界。在外视界$r_+$以外,均有球对称性,存在类时的基灵矢量;
    \item 若$M^2=Q^2$,此时上述两个奇性合二为一,这种情况被称为极端莱斯纳-努德斯特伦黑洞。第二章所提到的Wald的测试粒子方法\citep{wald1974gedanken},所采用的就是此种情况下的莱斯纳-努德斯特伦黑洞来验证弱宇宙监督假设。
\end{enumerate}
\subsection{反德西特时空下的莱斯纳-努德斯特伦黑洞}
式\eqref{eq: RNmetric}给出了零曲率空间下的莱斯纳-努德斯特伦黑洞的度规。在反德西特时空中,莱斯纳-努德斯特伦黑洞的度规为
\begin{equation}\label{eq: RN-AdSmetric}
    ds^2=-f\left(r\right)dt^2+\frac{1}{f\left(r\right)}dr^2+r^2\left(d\theta^2+\sin^2\phi d\phi ^2\right)
\end{equation}
度规的逆
\begin{equation}\label{eq: RN-AdSinversmetric}
    \left(\frac{\partial }{\partial s}\right)^2=-\frac{1}{f\left(r\right)}\left(\frac{\partial }{\partial t}\right)^2+f\left(r\right)\left(\frac{\partial }{\partial r}\right)^2+\frac{1}{r^2}\left[\left(\frac{\partial }{\partial \theta}\right)^2+\frac{1}{\sin ^2 \phi}\left(\frac{\partial }{\partial \phi}\right)^2\right]
\end{equation}
其中$f\left(r\right)$
\begin{equation}\label{eq: f(r)}
    f\left(r\right)=1-\frac{2M}{r}+\frac{Q^2}{r^2}+\frac{r^2}{l^2}
\end{equation}
其中$l$反德西特半径,它和宇宙学常数的关系为$\Lambda=-3/l^2$,参数$M$和$Q$,可以被理解为黑洞的质量和黑洞携带和电荷。
规范势可以写为
\begin{equation}\label{eq: vectorpotion}
    A=A_t\left(r\right)dt=-\frac{Q}{r}dt
\end{equation}

采用测试粒子的方法来验证对于反德西特时空下的莱斯纳-努德斯特伦黑洞弱宇宙监督假设是否成立,首先需要根据反德西特时空下的莱斯纳-努德斯特伦黑洞的度规式\eqref{eq: RN-AdSmetric},来讨论出极端反德西特时空下的莱斯纳-努德斯特伦黑洞的视界半径,此时令$ f(r)=0 $,会发现和常曲率空间下的莱斯纳-努德斯特伦黑洞不同的是,由于$r^2/l^2$这一项的存在,$ f(r)=0 $事实上为四次方程,这将带来四个根的存在。这并不说明反德西特时空下的莱斯纳-努德斯特伦黑洞存在四个视界半径,通过讨论发现,其中两根的值为虚数没有物理意义。反德西特时空下的莱斯纳-努德斯特伦黑洞的视界半径为
\begin{align}
    r_+=r + r_* \label{eq: rnr+} \\
    r_-=r-r_* \label{eq: rnr-} 
\end{align}
其中
\begin{align*}
    &r = \frac{1}{2} \gamma \\
    &r_* = \frac{1}{2} \sqrt{(\frac{4 l^2
    M}{\gamma}-\gamma^2-\frac{2 l^2}{3})}  \\
    &\alpha,\ \beta,\ \gamma \text{的值分别为} \\
    &\alpha =12 l^2 Q^2+l^4 \\ 
    &\beta =2 l^6 + 108 l^4 M^2 - 72 l^4 Q^2 \\
    &\gamma = \sqrt{\frac{\sqrt[3]{\sqrt{\beta^2-4 \alpha^3}+\beta}}{3\sqrt[3]{2}}+\frac{\sqrt[3]{2} \alpha}{3 \sqrt[3]{\sqrt{\beta^2-4\alpha^3}+\beta}}-\frac{2 l^2}{3}} 
\end{align*}

\subsection{反德西特时空下的莱斯纳-努德斯特伦黑洞存在条件}
观察解的形式发现,对比零曲率空间下的莱斯纳-努德斯特伦黑洞,得到反德西特时空下的极端莱斯纳-努德斯特伦黑洞为$r_+=r_-$,反德西特时空下的莱斯纳-努德斯特伦黑洞存在条件应为
\begin{equation}\label{eq: rnadsconuneq}
    \beta^2-4\alpha^3 \geq 0
\end{equation}
求解式\eqref{eq: rnadsconuneq},得到关于质量参数$M$、电荷参数$Q$和反德西特半径$l$的关系式\eqref{eq: rnadscondition}
\begin{equation}\label{eq: rnadscondition}
    M \geq \frac{\sqrt{\delta } (\delta +3) l}{3 \sqrt{6}} \qquad\qquad  \delta= \sqrt{\frac{12 Q^2}{l^2}+1}-1 
\end{equation}
还需要式\eqref{eq: rnadscondition}使得式\eqref{eq: suibian}得到满足
\begin{equation}\label{eq: suibian}
    \frac{\sqrt[3]{\sqrt{\beta^2-4 \alpha^3}+\beta}}{3\sqrt[3]{2}}+\frac{\sqrt[3]{2} \alpha}{3 \sqrt[3]{\sqrt{\beta^2-4\alpha^3}+\beta}}-\frac{2 l^2}{3} \geq 0
\end{equation}
简单计算验证,发现式\eqref{eq: suibian}得到满足。

以上的讨论是为了获得在反德西特时空下的极端莱斯纳-努德斯特伦黑洞的信息,通过比较前文中零曲率空间下的莱斯纳-努德斯特伦黑洞,我们可以知道反德西特时空下的极端莱斯纳-努德斯特伦黑洞的视界半径,将式\eqref{eq: rnadsconofmass}代入式\eqref{eq: rnr+}和式\eqref{eq: rnr-}
\begin{equation}\label{eq: rnadsconofmass}
    M=\frac{\sqrt{\delta } (\delta +3) l}{3 \sqrt{6}}
\end{equation}
反德西特时空下的极端莱斯纳-努德斯特伦黑洞的视界半径为
\begin{equation}\label{eq: rnadshorizonex}
    r=r_+=r_-=l\sqrt{\frac{\delta}{6}} 
\end{equation}

尝试创造一个违反式\eqref{eq: rnadscondition}的时空,通过向黑洞中丢入一个满足特殊条件的粒子,可以让式\eqref{eq: rnadscondition}取等号。假设此黑洞捕获了一个有着能量$E$,电荷为$e$,无自旋的粒子。因此,最后的物体具有电荷$\left(e+Q\right)$,质量不大于$\left(m+E\right)$。测试粒子需要满足$e\ll Q$,$E\ll M$的条件。简单计算得出,若末态时空不能保持为反德西特时空下的极端莱斯纳-努德斯特伦时空,那么式\eqref{eq: rnadscondition}成为
\begin{equation}\label{eq: rnadstestcon}
    M+E < \frac{\sqrt{\delta } (\delta +3) l}{3 \sqrt{6}} \quad\quad\quad\quad  \delta= \sqrt{\frac{12 \left(Q+e\right)^2}{l^2}+1}-1
\end{equation}
由于$e\ll Q$,$E\ll M$,式\eqref{eq: rnadstestcon}平方,泰勒展开,得到如果测试粒子破坏反德西特时空下的极端莱斯纳-努德斯特伦黑洞$E$需要满足的条件
\begin{equation}\label{eq: rnadsdiscon}
    E < \frac{Q e \left(3+\delta\right)}{3M} \qquad\qquad  \delta= \sqrt{\frac{12 Q^2}{l^2}+1}-1
\end{equation}

本次思想实验的目标就是,尝试让黑洞捕获一个满足式\eqref{eq: rnadsdiscon}条件的小物体。如此,最终的时空必然会出现违反弱宇宙监督假设的情况\citep{wald1974gedanken}。
\subsection{测试粒子的轨道}
Carter非常详尽的研究了在克尔-纽曼几何下的粒子运动\cite{carter1968global}。有着静止质量$M$和电荷$Q$的带电粒子的运动方程为
\begin{equation}
    \frac{D^2x^\mu}{D s^2}=\frac{d^2x^\mu}{ds^2}+\Gamma^{\mu}_{\rho\sigma }\frac{dx^\rho}{ds}\frac{dx^\sigma}{ds}=\frac{e}{m}F^{\mu\nu }\frac{dx_\nu}{ds}
\end{equation}
粒子的能量$E$和沿对称轴的角动量$L$分量由以下给出,由Wald的方法\cite{wald1974gedanken},根据式()
\begin{align}
    -E&=p_t=\frac{\partial \mathcal{L}}{\partial \dot{t}}=Mg_{tt}\frac{dx^t}{ds}+QA_t \\
    L&=p_\phi=\frac{\partial \mathcal{L}}{\partial \dot{\phi}}=0
\end{align}

在之前的讨论中,我们现在试图找到一个可以进入反德西特时空下的极端莱斯纳-努德斯特伦黑洞的粒子轨道,并且满足式\eqref{eq: rnadsdiscon}的条件。

根据式(第二章写wald的方法的时候搞进去),得到
\begin{equation}\label{eq: xiaren}
    -M^2=g^{tt}\left(-E-eA_t\right)^2+g^{rr}p_r^2+g^{\theta \theta}p_\theta ^2
\end{equation}
通过解式\eqref{eq: xiaren},由于$dt/ds>0$,只保留其中一个解
\begin{equation}
    E=-eA_t -\frac{1}{g^{tt}}\sqrt{-M^2g^{tt}+p_r^2-g^{tt}g^{\theta \theta}p_{\theta}^2}
\end{equation}
由此,根据时间的方向性,我们得到
\begin{equation}
    E>-eA_t=\frac{eQ}{r}
\end{equation}
此时我们讨论的为反德西特时空下的极端莱斯纳-努德斯特伦黑洞,由式\eqref{eq: rnadshorizonex}和式\eqref{eq: rnadscondition}取等号,得到
\begin{equation}\label{eq: rnadsinter}
    E> \frac{Qe\left(3+\delta\right)}{3M} \qquad\qquad  \delta= \sqrt{\frac{12 Q^2}{l^2}+1}-1
\end{equation}
任何可以进入黑洞的粒子必循满足式\eqref{eq: rnadsinter},从而式\eqref{eq: rnadsdiscon}被违背。保证了违反弱宇宙监督假设的情况不能获得。

\section{双曲黑洞}
类似于莱斯纳-努德斯特伦黑洞,双曲黑洞也属于克尔-纽曼黑洞中的一员,反德西特时空下的双曲带电黑洞和反德西特时空下的莱斯纳-努德斯特伦黑洞的度规非常相似,见下式\citep{cai1999topological}
\begin{align}\label{eq: general}
    ds^2=&-\left(k-\frac{8\pi M}{\omega_2 r}+\frac{16 \pi^2 Q^2}{\omega_2^2r^2}+\frac{r^2}{l^2}\right)dt^2 \notag \\
    &+\left(k-\frac{8\pi M}{\omega_2 r}+\frac{16\pi^2 Q^2}{w_2^2 r^2}+\frac{r^2}{l^2}\right)^{-1}dr^2+r^2d\Sigma_2^2
\end{align}
其中,$Q$和$M$为黑洞的电荷与质量,$\omega_2$为视界边界超曲面$\Sigma_2$的的面积。$d\Sigma_2^2$为曲率为常数$2k$二维超曲面$\Sigma_2$的线元;
\begin{equation}\label{eqs: differentsituation}
    d\Sigma^2_2=\begin{cases}
        d\theta^2+\sin ^2 \theta d\phi^2 \qquad \text{对于} \quad k=1, \\
        d\theta^2+ \theta^2 d\phi^2 \qquad \text{对于} \quad k=0, \\
        d\theta^2+\sinh ^2 \theta d\phi^2 \qquad \text{对于} \quad k=-1.
    \end{cases}
\end{equation}
在式\eqref{eqs: differentsituation}中,使用了事件视界的二维超曲面的常曲率分别为$1$, $0$和$-1$的坐标,这种做法依然保留了一般性。并且在$k=1$的情况下,式\eqref{eq: general}变成了反德西特时空下的莱斯纳-努德斯特伦黑洞,事件视界为$S^2$。视界边界为双曲面$T^2$在$k=-1$的时候,有趣的是即使质量$M$为负数的时候,依然存在视界边界,这种负质量黑洞可以在正常的引力坍缩中形成\citep{mann1997black,smith1997formation}。这种负质量情况会带来很多有意思的情况,但是对于采用测试粒子验证弱宇宙监督假设,我们通过分情况讨论发现,在此情况下不需要特别关注负质量情况。

\subsection{反德西特时空下的双曲黑洞}

现在考虑反德西特时空下的不带电的双曲黑洞,根据式\eqref{eq: general}其度规为
\begin{equation}
    ds^2=-f\left(r\right)dt^2+f\left(r\right)^{-1}dr^2+r^2d\Sigma_{2}^2
\end{equation}
其中$f\left(r\right)$为
\begin{align}
    &f\left(r\right)=-1-\frac{2m}{r}-\frac{\Lambda r^2}{3}+\frac{Q^2}{r^2} \\
    &\Lambda = -3 l^{-2}
\end{align}
在Curry的文章中\citep{curry1991vacuum}给出了不带电情况下反德西特时空下的双曲黑洞的视界半径,借助于此我们可以更清楚的了解双曲黑洞带来的负质量情况是如何影响视界半径的,表\ref{tab: hyadswithoutcharge}为Curry给出的视界半径。
其中$a$和$\eta$的定义为\begin{align*}
    &a \equiv \left[-\frac{3m}{\Lambda}+\left(\frac{9m^2}{\Lambda^2}-\frac{k^3}{\Lambda^3}\right)^{1/2}\right]^{1/3} \\
    &\cos \eta =-\frac{3m}{\Lambda}\left(\frac{\Lambda}{k}\right)^{3/2}
\end{align*}

\begin{table}[htb]
    \centering
    \begin{minipage}[t]{0.8\linewidth} % 如果想在表格中使用脚注,minipage是个不错的办法
    \caption{反德西特时空下的双曲黑洞视界半径} 
      \label{tab: hyadswithoutcharge}
      \begin{tabularx}{\linewidth}{lX}
        \toprule[1.5pt]
        {\heiti $\Lambda$} & {\heiti $k=-1$} \\\midrule[1pt]
        $<-1/9m^2$ & $r_1=a$ \\
        $-1/9m^2$ &  $r_1=6m, \ m>0$,  \\
        & $r_2=r_3=-3m , \ m<0$.
                      \\
        $-1/9m^2<\Lambda<0$ &  $r_1=2\sqrt{-\frac{1}{\Lambda}}\cos \left(\frac{1}{3}\eta \right), \ m>0$, \\ 
        & 或者 \\
        & $r_2=2\sqrt{-\frac{1}{\Lambda}} \cos \left(\frac{1}{3}\eta +\frac{4}{3}\pi\right), \ m>0$ \\
        & $r_3=a, \ m<0$ \\
        \bottomrule[1.5pt]
      \end{tabularx}
    \end{minipage}
  \end{table}



\section{反德西特时空下的双曲带电黑洞}
式\eqref{eq: general}给出了,反德西特时空下的双曲带电黑洞的度规。取$k=-1$后,度规为
\begin{equation}
    ds^2=-f\left(r\right)dt^2+\frac{1}{f\left(r\right)}dr^2+r^2d\Sigma^2_2
\end{equation}
度规的逆为
\begin{equation}
    \left(\frac{\partial }{\partial s}\right)^2=-\frac{1}{f\left(r\right)}\left(\frac{\partial }{\partial t}\right)^2+f\left(r\right)\left(\frac{\partial }{\partial r}\right)^2+\frac{1}{r^2}\left[\left(\frac{\partial }{\partial \theta}\right)^2+\frac{1}{\sinh ^2 \phi}\left(\frac{\partial }{\partial \phi}\right)^2\right]
\end{equation}
其中
\begin{equation}
    f\left(r\right)=-1-\frac{2m}{r}-\frac{\Lambda r^2}{3}+\frac{Q^2}{r^2}
\end{equation}
规范势为
\begin{equation}
    A=A_t\left(r\right)dt=-\frac{Q}{r}dt
\end{equation}
和反德西特时空下的莱斯纳-努德斯特伦黑洞相似(参见式\eqref{eq: RN-AdSmetric}),参数$m$和$Q$被理解为黑洞的质量参数和电荷参数。经过同样的讨论我们可以得出来反德西特时空下的双曲带电黑洞的视界半径,依然舍去两个虚数的解,保留有物理意义的实数解。
\begin{align}
    r_+=r + r_* \label{eq: hyr+} \\
    r_-=r-r_* \label{eq: hyr-} 
\end{align}
其中
\begin{align*}
    r=&\frac{1}{2}\sqrt{-\frac{2}{\Lambda}-\frac{3 \times 2^{\frac{1}{3}}\alpha}{\Lambda \left(\beta+\sqrt{\beta^2-4 (9\alpha)^3} \right)^{\frac{1}{3}}}-\frac{1}{3 \times 2^{\frac{1}{3}}\Lambda}\left(\beta +\sqrt{\beta^2-4 \left(9\alpha\right)^3}\right)^{\frac{1}{3}}} \\
    r_*=&\frac{1}{2}\sqrt{-\frac{12m}{\Lambda \sqrt{-\frac{2}{\Lambda}-\frac{3 \times 2^{\frac{1}{3}} \alpha}{\Lambda \left(\beta+\sqrt{\beta^2 - 4 (9\alpha)^3}\right)^{\frac{1}{3}}}-\frac{1}{3\times 2^{\frac{1}{3}}}\left(\beta+\sqrt{\beta^2-4(9 \alpha)^3}\right)^{\frac{1}{3}}}}} \notag \\
    &\overline{+\frac{3 \times 2^{\frac{1}{3}}\alpha}{\Lambda \left(\beta+\sqrt{\beta^2-4 (9\alpha)^3} \right)^{\frac{1}{3}}}+\frac{1}{3 \times 2^{\frac{1}{3}}\Lambda}\left(\beta +\sqrt{\beta^2-4 \left(9\alpha\right)^3}\right)^{\frac{1}{3}}-\frac{4}{\Lambda}} \\
    &\alpha,\ \beta,\ \gamma \text{的值分别为} \notag \\
    \alpha =& 1-4\Lambda  Q^2    \notag \\
    \beta =&-972 \Lambda  m^2-648 \Lambda Q^2-54 \notag 
\end{align*}

采用与前一节相同的办法,首先找到反德西特时空下的双曲带电黑洞的存在条件,通过让其捕获测试粒子来尝试得到“裸奇点”,从而得到违反弱宇宙监督假设的结果。德西特时空下的双曲带电黑洞的存在条件应为
\begin{equation}\label{eq: hycon}
    \beta^2-4 \left(9\alpha\right)^3 \geq 0
\end{equation}
求解式\eqref{eq: hycon},得到关于质量参数$M$、电荷参数$Q$和宇宙学常数$\Lambda$的关系式\eqref{eq: hyadscondition}
\begin{equation}\label{eq: hyadscondition}
    m \geq -\frac{1}{12} \Lambda  \sqrt{-\frac{2}{\Lambda }-\frac{2 \sqrt{1-4 \Lambda  Q^2}}{\Lambda }} \left(\frac{4}{\Lambda }-\frac{2 \sqrt{1-4 \Lambda 
    Q^2}}{\Lambda }\right)
\end{equation}
如果取$Q=0$会发现此时的结果和表\ref{tab: hyadswithoutcharge}的视界半径一致。把宇宙学常数$\Lambda$换成反德西特半径$l$,这样可以和反德西特时空下的莱斯纳-努德斯特伦黑洞对比。
\begin{equation}\label{eq: hyadsconditionreal}
    m\geq \frac{l\left(\delta-3\right)\sqrt{\delta}}{3\sqrt{6}} \qquad \qquad \delta=\sqrt{1+12Q^2/l^2}+1
\end{equation}

再次情况下,我们也可以得出反德西特空间下极端双曲带电黑洞的视界半径为
\begin{equation}\label{eq: hyadsextramhorizon}
    r=r_+=r_-=\sqrt{\frac{\delta}{6}}l
\end{equation}

对于反德西特时空下的莱斯纳-努德斯特伦黑洞,此时我们可以计算出黑洞捕获测试粒子后可以使得末态时空被违反的条件式\eqref{eq: rnadsdiscon},但是考虑到在反德西特空间下双曲带电黑洞的负质量情况令人担忧,因为没有理由要求式\eqref{eq: hyadscondition} $\left(4/\Lambda -2 \sqrt{1-4 \Lambda Q^2}/\Lambda \right)$这一项始终大于零,那么负质量情况的讨论就不可避免,接下来的两节将分别讨论黑洞的质量参数$m$大于零和小于零的两种情况,最后我们发现即使一开始不区分情况讨论,只对过程中不等号的方向有影响,而对最终的结论并无影响。
\subsection{正质量情况}
此时我们考虑正质量情况,通过向黑洞中丢入一个满足特殊条件的粒子,可以让式\eqref{eq: hyadsconditionreal}取等号。假设此黑洞捕获了一个有着能量$E$,电荷为$e$,无自旋的粒子。因此,最后的物体具有电荷$\left(e+Q\right)$,质量不大于$\left(m+E\right)$。测试粒子需要满足$e\ll Q$,$E\ll m$的条件。简单计算得出,若末态时空不能保持为反德西特时空下的极端双曲黑洞,那么式\eqref{eq: hyadsconditionreal}成为
\begin{equation}\label{eq: hyadsputtestparticle}
    m+E < \frac{\sqrt{\delta } (\delta -3) l}{3 \sqrt{6}} \quad\quad\quad\quad  \delta= \sqrt{\frac{12 \left(Q+e\right)^2}{l^2}+1}+1
\end{equation}
不等式两端平方以后,再取泰勒展开只保留一阶项。那么可以得到
\begin{equation}\label{eq: hyadsdiscon}
    E < \frac{Q e \left(\delta-3\right)}{3 m} \qquad \qquad \delta=\sqrt{1+12Q^2/l^2}+1
\end{equation}

粒子的轨道根据式(第二章搞进去),其方程和式\eqref{eq: xiaren}相同,同样由于$dt/ds>0$,和时间的方向性,我们可以得到
\begin{equation}\label{eq: hyadstestcon}
    E>-eA_t=\frac{eQ}{r}
\end{equation}
将式\eqref{eq: hyadsextramhorizon}和式\eqref{eq: hyadsconditionreal}取等号,代入式\eqref{eq: hyadstestcon}中,可以得到
\begin{equation}\label{eq: woshizaibuzhidaogaogeshale}
    E > e\frac{Q}{r}=\frac{Q e \left(\delta-3\right)}{3 m} 
\end{equation}
对比式\eqref{eq: woshizaibuzhidaogaogeshale}和式\eqref{eq: hyadsdiscon},任何可以进入黑洞的粒子就不能满足摧毁黑洞时空的条件,由此弱宇宙监督假设再次得到了保证。
\subsection{负质量情况}
对于负质量情况,首先需要明确的问题为黑洞捕获了测试粒子以后的质量不大于$m+E$,是否依然不大于零。很显然如果大于零的话,$E \ll m$的条件就不能得到满足。第二个为题在于对式\eqref{eq: hyadsputtestparticle}进行平方操作后不等号会反向。但是由于泰勒展开后存在$E m $项,不等号会再次反向。综上所述,在采用测试粒子验证弱宇宙监督假设的情况下,特别考虑负质量问题并无必要。

\section{本章小结}
本章通过使黑洞捕获一个小物体来创造出一个“裸奇点”的情况,通过测地线方程在并没有真实求解粒子运动轨道的情况下讨论了可以被黑洞捕获小物体的能量和电荷条件,由于静电斥力,可以破坏极端黑洞的粒子都不满足这个条件。

尽管我们可以任意的接近极端黑洞的情况但永远不能进入,测试粒子总会“错过”黑洞。
