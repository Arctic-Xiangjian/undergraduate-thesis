\chapter{外文资料的调研阅读报告或书面翻译}

\title{英文资料的中文标题}

{\heiti 摘要:} 本章为外文资料翻译内容。如果有摘要可以直接写上来,这部分好像没有
明确的规定。

\section{单目标规划}
北冥有鱼,其名为鲲。鲲之大,不知其几千里也。化而为鸟,其名为鹏。鹏之背,不知其几
千里也。怒而飞,其翼若垂天之云。是鸟也,海运则将徙于南冥。南冥者,天池也。
\begin{equation}\tag*{(123)}
 p(y|\mathbf{x}) = \frac{p(\mathbf{x},y)}{p(\mathbf{x})}=
\frac{p(\mathbf{x}|y)p(y)}{p(\mathbf{x})}
\end{equation}

吾生也有涯,而知也无涯。以有涯随无涯,殆已!已而为知者,殆而已矣!为善无近名,为
恶无近刑,缘督以为经,可以保身,可以全生,可以养亲,可以尽年。

\subsection{线性规划}
庖丁为文惠君解牛,手之所触,肩之所倚,足之所履,膝之所倚,砉然响然,奏刀騞然,莫
不中音,合于桑林之舞,乃中经首之会。

文惠君曰:“嘻,善哉!技盖至此乎?”庖丁释刀对曰:“臣之所好者道也,进乎技矣。始臣之
解牛之时,所见无非全牛者;三年之后,未尝见全牛也;方今之时,臣以神遇而不以目视,
官知止而神欲行。依乎天理,批大郤,导大窾,因其固然。技经肯綮之未尝,而况大坬乎!
良庖岁更刀,割也;族庖月更刀,折也;今臣之刀十九年矣,所解数千牛矣,而刀刃若新发
于硎。彼节者有间而刀刃者无厚,以无厚入有间,恢恢乎其于游刃必有余地矣。是以十九年
而刀刃若新发于硎。虽然,每至于族,吾见其难为,怵然为戒,视为止,行为迟,动刀甚微,
謋然已解,如土委地。提刀而立,为之而四顾,为之踌躇满志,善刀而藏之。”

文惠君曰:“善哉!吾闻庖丁之言,得养生焉。”


\subsection{非线性规划}
孔子与柳下季为友,柳下季之弟名曰盗跖。盗跖从卒九千人,横行天下,侵暴诸侯。穴室枢
户,驱人牛马,取人妇女。贪得忘亲,不顾父母兄弟,不祭先祖。所过之邑,大国守城,小
国入保,万民苦之。孔子谓柳下季曰:“夫为人父者,必能诏其子;为人兄者,必能教其弟。
若父不能诏其子,兄不能教其弟,则无贵父子兄弟之亲矣。今先生,世之才士也,弟为盗
跖,为天下害,而弗能教也,丘窃为先生羞之。丘请为先生往说之。”

柳下季曰:“先生言为人父者必能诏其子,为人兄者必能教其弟,若子不听父之诏,弟不受
兄之教,虽今先生之辩,将奈之何哉?且跖之为人也,心如涌泉,意如飘风,强足以距敌,
辩足以饰非。顺其心则喜,逆其心则怒,易辱人以言。先生必无往。”

孔子不听,颜回为驭,子贡为右,往见盗跖。

\subsection{整数规划}
盗跖乃方休卒徒大山之阳,脍人肝而餔之。孔子下车而前,见谒者曰:“鲁人孔丘,闻将军
高义,敬再拜谒者。”谒者入通。盗跖闻之大怒,目如明星,发上指冠,曰:“此夫鲁国之
巧伪人孔丘非邪?为我告之:尔作言造语,妄称文、武,冠枝木之冠,带死牛之胁,多辞缪
说,不耕而食,不织而衣,摇唇鼓舌,擅生是非,以迷天下之主,使天下学士不反其本,妄
作孝弟,而侥幸于封侯富贵者也。子之罪大极重,疾走归!不然,我将以子肝益昼餔之膳。”
