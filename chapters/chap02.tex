\chapter{宇宙监督假设}
史瓦西黑洞这样的类空间奇点并不是问题,因为任何内部的观察者都看不到奇点,一个人不会像跌入洞穴一样跌入史瓦西奇点。奇点不可避免,就像下周一肯定会到来一样。任何信息都不能从奇点传播到观察者。
\section{弱宇宙监督假设}
弱宇宙监督假设要求,奇点应在黑洞视界之后。其适用于黑洞内的时间型奇点,比如带电或旋转的黑洞内的奇点,因为根据黑洞的定义,关于奇点的信息不能影响外部世界。
\subsection{违反弱宇宙监督假设的尝试}
Choptuik证明了在非常特殊的初始条件下\citep{choptuik1993universality},可以形成一个裸奇点。但这样的粒子已经被弱宇宙监督假设排除,因为弱宇宙监督假设只要求在一般的初始条件下不形成裸奇点。

另一个违反弱宇宙监督假设的尝试为摧毁一个黑洞的视界,这也是本文采用的方法。Wald的论文\citep{wald1974gedanken}和之后的许多作品都证明了当黑洞越来越接近极值时,也就越来越难扔进可能破坏视界的粒子。

但是弱宇宙监督假设并不是那么牢不可破了。Santos 和他的合作者对于边界有特殊化学势的黑洞—Hovering 黑洞进行了数值分析\citep{horowitz2015hovering,crisford2017violating},似乎得出了和 Wald 相反的结论。
\subsection{测试粒子}
向极端黑洞中丢入一个测试粒子,使黑洞捕获其后最终的物体不在满足原时空的条件,从而导致裸奇点被破坏。并且由于只需要计算出黑洞的存在条件、粒子能够进入黑洞的条件和破坏视界时粒子需要满足的条件。这种方法在计算上也非常简单易行。

后来有研究称,如果 Wald 使用的测试粒子电荷满足一定的大小,那么在只考虑一阶修正项的情况下,是可以破坏黑洞视界的\citep{hubeny1999overcharging}。但是很快就被发现,在此情况下测试粒子背景时空的印象不能忽略\citep{hod2002cosmic,barausse2010test,colleoni2015overspinning,wald2018kerr,sorce2017gedanken},弱宇宙监督假设依然得到了保证。

\section{强宇宙监督假设}
强宇宙审查假说本质上要求未来应该是完全确定的,即使是在黑洞里。毕竟,一个观测者可以被送入一个任意大小的黑洞,如果他或她在旅行中幸存下来,就可以在黑洞内进行物理观测和实验。至少在原则上,人们可以在黑洞内部进行物理研究,而不必向外部时空的同事报告研究结果\citep{ong2020space}。或者其中一种简单的表述为真实时空是整体双曲时空。

关于强宇宙监督假设其他的标注需要借助“理想点”和TIP和TIF等概念,这些概念的定义可以在\citep{梁灿彬2006微分几何入门与广义相对论}中找到。
\begin{enumerate}
    \item 真实时空不存在局部裸的TIP和TIF;
    \item 真实时空的TIP和TIF的几何是非编时的。
\end{enumerate}

作为例子,可以考察球对称坍缩导致的史瓦西奇性,这种类空的奇性是符合强宇宙监督假设的,因为不但其在事件视界之内,更因为它根本不是编时的。反之莱斯特-努德斯特伦黑洞,其奇性为类时的。幸好这种时空是不稳定的,只要略受微扰就会在其柯西视界附近发展出一个类空(或类光)奇性,因而不再为局部裸。