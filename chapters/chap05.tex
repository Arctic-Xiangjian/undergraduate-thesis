\chapter{总结与展望}
\section{总结}
本文采用了测试粒子的方法对三种黑洞进行了弱宇宙监督假设的验证,静电斥力保证了假设不被违反。大自然在极端黑洞情形下显示了对黑洞的保护手段,令人瞩目,简直象是在有预谋地阻止裸奇点的出现,这是支持宇宙监督假设的一类很直接的证据。
\section{展望}
真如前文所提到的那样,我们不能把所有希望都寄托于一个完整的量子引力或更宏伟的终极理论 (theory of everything) 中。甚至宇宙监督假设的违反也可以提供一些关于量子引力的暗示,一些关于量子引力的猜想仍然保留着宇宙审查假设。

在黑洞概念问世之初,英国物理学家埃丁顿曾认为大自然将会阻止黑洞出现,就像宇宙监督假设的支持者们认为大自然将会阻止裸奇点出现一样;在量子力学问世之初, 爱因斯坦曾像霍金那样猜测过“上帝”的喜好,认为“上帝”不会掷骰子,结果爱因斯坦错了。历史是否会在宇宙监督假设这里重演,还是另谋篇章?我们不得而知。既然奇点应该携带有关量子引力的信息,那么来自奇点的任何信号都应该受到欢迎。它们是机会之窗,让我们进入至今还无法到达的量子引力领域。