\begin{cabstract}[黑洞;反德西特时空;宇宙监督;高斯-博尼特引力]
    奇点定理指出,如果爱因斯坦的广义相对论正确,物质满足一定的能量条件,那么将一定存在一个奇点,在这一点处时空曲率是发散的,时间在此处起始或终结。如果奇点不被黑洞视界所掩盖,也就是变得“赤裸”,那么广义相对论中就会出现不确定性。为了解决这个问题,一开始人们希望在任何过程中都不可能产生奇点,但是很快就构造出了一些反例。所以人们的期望就变为:对于一般的初始条件,裸奇点不可能出现。这就是宇宙监督假设。弱宇宙监督假设可以表述为,奇点必须被隐藏在黑洞的事件视界之后。一个很自然的验证想法就是通过向极端黑洞里面放入新粒子,来破坏黑洞的视界,从而获得一个“裸奇点”。1968 年,Wald 验证了克尔-纽曼黑洞的情况。在第三章,我们验证了在渐进反德西特时空中黑洞的例子,分别为莱斯特-努德斯特伦黑洞和双曲黑洞,双曲黑洞中负质量的情形对验证过程的影响非常小。第四章,我们讨论了对于爱因斯坦引力的拓展,四维时空的高斯-博尼特引力,这种以往被认为平凡的模型在新的方法下可以对其进行进一步研究,在这种时空下弱宇宙监督假设依然得到满足。
\end{cabstract}

\begin{eabstract}[Black hole;Anti de Sitter;Cosmic Censorship;Gauss-Bonnett gravity]
    The Singularity Theorem states that the Einstein's general theory of relativity, if matter satisfies certain energy conditions, then there must be a singularity at which the curvature of space-time diverges, and at which time begins or ends. If the singularity is not obscured by the black hole's event horizon, that is, becomes "naked", then there is indetermination in general relativity. To solve this problem, it was initially hoped that singularities could not be produced in any process, but some counterexamples were soon constructed. Hence the expectation becomes that for general initial conditions, naked singularities are not possible to generate. That's the Cosmic Censorship Conjecture. The Weak Cosmic Censorship Conjecture is that the singularity must be hidden behind the event horizon. There is a natural idea to test it, trying to create a "naked singularity" by throwing a test particles into an extreme black hole that would disrupt its event horizon. In 1968, Wald verified the condition of a Kerr - Newman black hole. In Chapter 3, we verify the RN-AdS black hole and hyperbolic charged AdS black hole. The negative mass condition in hyperbolic charged AdS black hole has a limit influence on the verification process. In Chapter 4, we discuss the extension of Einstein's gravity, the Gauss-Bonnett gravity in four-dimensional space-time. This model was considered trivial in the past, can be further studied in a new way. In this space-time, the weak cosmic supervision hypothesis is still satisfied.

\end{eabstract}

\makecover
\tableofcontents
