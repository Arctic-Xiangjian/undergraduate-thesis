\begin{cabstract}[黑洞;反德西特时空;宇宙监督;高斯-博尼特引力]
    我们使用思想实验的方法,在计算出黑洞的存在条件的前提下,考虑极端黑洞的情形,来验证弱宇宙监督假设。假设若极端黑洞和测试粒子的结合体的参数不再满足黑洞存在条件,计算出测试粒子需要的条件,随后使用运动学方程,再次计算可以被黑洞捕获的粒子需要满足的条件。通过对比以上结果,发现两条件之间没有交集。本文研究的黑洞可以分为三类,分别为爱因斯坦引力中的渐近反德西特下的莱斯特-努德斯特伦黑洞和双曲视界面带电黑洞以及四维新颖爱因斯坦-高斯-博尼特引力中的渐近德西特带电黑洞。在双曲视界面带电黑洞中,负质量情况经过了特别讨论后,和正质量情况下结论依然相同。由于静电斥力的影响,以上极端黑洞均不能捕获可以破坏黑洞视界的测试粒子,弱宇宙监督假设依然可靠。
\end{cabstract}

\begin{eabstract}[Black hole;Anti de Sitter;Cosmic Censorship Conjucture;Gauss-Bonnett gravity]
    We use the method of thought experiment to test the Weak Cosmic Censorship Conjucture by considering the extreme black hole under the premise of calculating the existence condition of black hole. Assuming that the parameters of the combination of the extreme black hole and the test particle no longer meet the conditions for the existence of the black hole, the conditions required for the test particle are calculated. Then, the kinematics equation is used to calculate the conditions needed for the particles that can be captured by the black hole again.By comparing the above results, it is found that there is no intersection between the two conditions. The black holes studied in this paper can be divided into three categories: the Reissner-Nordström black hole and the charged hyperbolic event horizon black hole under the asymptotically anti-de Siter in Einstein's gravity and the asymptotically charged de Siter black hole in the four-dimensional novel Einstein-Gauss-Bonnet gravity. In the hyperbolic event horizon charged black hole, after special discussion, the negative mass case and the positive mass case still come to the same conclusion. Due to the effect of electrostatic repulsion, none of the above extreme black holes can trap the test particles that can destroy the event horizon of the black hole, and the hypothesis of Weak Cosmic Censorship Conjucture is still reliable.

\end{eabstract}

\makecover
\tableofcontents
