\begin{cabstract}[\LaTeX 模板;本科生毕业论文;四川大学]
    普遍认为,物体的引力坍缩的结果总是一个黑洞(即,“裸奇点”不可能产生),通过使用三个在渐进反德西特时空中黑洞的例子,尝试摧毁其视界来获得“裸奇点”从而违反弱宇宙监督假设。其中一个黑洞为爱因斯坦引力的拓展,四维时空的高斯-博尼特引力,对其进行测试粒子验证发现测试粒子的轨道同引力的形式并无关系。双曲黑洞中负质量情形同样不能影响极端黑洞保持时空存在的条件。
\end{cabstract}

\begin{eabstract}[\LaTeX{} Template;Sichuan University;Bachelor Dissertations]

\end{eabstract}

\makecover
\tableofcontents
