\begin{cabstract}[黑洞;反德西特时空;宇宙监督假设;爱因斯坦-高斯-博尼特引力]
    我们采用思想实验的方法,通过考虑向带电极端黑洞丢入测试粒子,来验证弱宇宙监督假设。首先由极端黑洞捕获测试粒子后参数不再满足黑洞存在条件的假设,计算出测试粒子能量需要满足的条件,随后根据粒子运动中的能量守恒条件,计算可以抵达黑洞视界的粒子需要满足的条件。通过对较以上结果,发现两者之间没有交集。本文研究三类黑洞,分别为爱因斯坦引力中的渐近反德西特莱斯特-努德斯特伦黑洞和双曲视界面带电黑洞,以及四维新颖爱因斯坦-高斯-博尼特引力中的渐近反德西特带电黑洞。对于双曲视界面带电黑洞中,我们特别讨论了特殊的负质量情况,发现和正质量情况下结论相同。对于四维新颖爱因斯坦-高斯-博尼特引力中的渐近反德西特带点黑洞,我们发现通过电荷参数的重新选取,情况和莱斯特-努德斯特伦黑洞非常类似。由于静电斥力的影响,以上极端黑洞均不能捕获可以破坏黑洞视界的测试粒子,弱宇宙监督假设依然可靠。
\end{cabstract}

\begin{eabstract}[Black hole;Anti de Sitter;Cosmic Censorship Conjucture;Einstein-Gauss-Bonnett gravity]
    We use a gedanken experiment to test the weak cosmic censorship conjucture by considering throwing test particles into a charged extremal black hole. After the extremal black hole captures the test particle, hypothesize the parameters no longer meet the existence condition of the black hole, and the conditions that the energy of the test particle needs to meet are calculated. Then, according to the energy conservation condition in the motion of the particle, the conditions that the particle can reach the event horizon need to meet are calculated. According to the above results, there is no intersection between the two. Three types of black holes are studied in this paper. They are RN-AdS black hole and charged hyperbolic black hole in Einstein's gravity, and AdS black hole in 4-dimensional novel Einstein-Gauss-Bonite gravity. For the hyperbolic charged black hole, we discuss the special negative mass case and find the same conclusion as the positive mass case. For the AdS black hole in the 4-dimensional Einstein-Gauss-Bonite gravity, we find that the situation is very similar to RN-AdS black hole through the re-selection of charge parameters. Due to the effect of electrostatic repulsion, none of the above extremal black holes can trap the test particles that can destroy the event horizon of the black hole, and the hypothesis of weak cosmic supervision is still reliable.

\end{eabstract}

\makecover
\tableofcontents
